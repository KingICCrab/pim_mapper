%!TEX program = xelatex
\documentclass[UTF8]{ctexart}
\usepackage{geometry}
\usepackage{enumitem}
\usepackage{hyperref}

\geometry{a4paper, margin=1in}

\title{ILP 筛选能力验证报告}
\author{PIM Optimizer Team}
\date{\today}

\begin{document}

\maketitle

\section{实验目标}
验证 ILP 模型是否能够有效地“筛选”搜索空间,即识别出高质量的候选映射方案,以便后续使用 Trace 模拟器进行精确验证。

\section{实验方法}
\begin{enumerate}
    \item \textbf{生成候选池}:使用 Gurobi 的 \texttt{PoolSearchMode} 生成前 200 个 ILP 解。
    \item \textbf{验证}:对所有 200 个解运行 \texttt{FastTraceGenerator},以获取“真实”的代价(行激活数,Row Activations)。
    \item \textbf{相关性分析}:对比 ILP 排名与真实排名。
\end{enumerate}

\section{限制与修改}
由于 **Gurobi License 限制(2000 个变量)**,无法运行完整的 ILP 模型。为了进行该实验,模型被大幅简化(Stripped):
\begin{itemize}
    \item \textbf{禁用}:行激活(Row Activation)约束(这是主要优化指标)。
    \item \textbf{禁用}:复用跟踪(Reuse Tracking)约束。
    \item \textbf{简化}:目标函数从“延迟(Latency)”更改为“总计算量(Total Compute)”(即最大化并行度)。
    \item \textbf{近似}:分段线性(PWL)函数减少为 1 段。
\end{itemize}

\section{实验结果}
\begin{itemize}
    \item \textbf{执行情况}:成功使用简化模型(约 313 个变量)生成了 200 个解。
    \item \textbf{筛选性能}:
    \begin{itemize}
        \item \textbf{真实最优解(True Best Solution,最低行激活数)} 出现在 \textbf{ILP 排名的第 99 位}。
        \item ILP 认为最好的前 10 个解(针对并行度优化)在行激活数(内存性能)上表现不佳。
    \end{itemize}
\end{itemize}

\section{结论}
实验证实了 \textbf{完整 ILP 模型的必要性}。
\begin{itemize}
    \item 简化模型(仅优化计算)与 Trace 指标(优化内存)之间的相关性很差。
    \item 这一负面结果验证了完整模型设计的合理性:必须包含那些导致变量数量激增的复杂约束(如行激活和复用跟踪),才能准确捕捉内存行为。
\end{itemize}

\section{当前状态}
\begin{itemize}
    \item \texttt{src/} 中的源代码已 \textbf{回滚} 到原始状态(完整模型)。
    \item 验证脚本 \texttt{verify\_ilp\_filtering\_capability.py} 已保留,以备将来在拥有完整 License 时使用。
\end{itemize}

\end{document}
