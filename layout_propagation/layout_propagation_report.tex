\documentclass[UTF8,a4paper,11pt]{ctexart}
\usepackage{geometry}
\usepackage{graphicx}
\usepackage{booktabs}
\usepackage{amsmath}
\usepackage{algorithm}
\usepackage{algpseudocode}
\usepackage{listings}
\usepackage{xcolor}
\usepackage{float}
\usepackage{hyperref}

\geometry{left=2.5cm,right=2.5cm,top=2.5cm,bottom=2.5cm}

\title{\textbf{自动化数据布局传播与优化:\\ 实验方法与结果分析}}
\author{实验报告}
\date{\today}

\begin{document}

\maketitle

\begin{abstract}
本报告详细介绍了 \texttt{layout\_propagation} 模块中的实验方法、结果数据及深入分析。我们针对深度神经网络加速器中普遍存在的“数据布局不匹配”问题,提出了一套基于代价模型的自动化布局传播框架。实验结果表明,该框架在处理具有复杂分支结构的现代网络(如 ResNet50)时,能够将总访存代价降低约 50\%,显著优于传统的固定布局策略。
\end{abstract}

\section{实验方法 (Methodology)}

\subsection{问题背景}
在深度学习加速器中,不同的算子(Operator)对数据布局有不同的偏好。例如,卷积核可能偏好 $K \times C \times H \times W$ 以最大化复用,而脉动阵列可能需要分块的布局(Blocked Layout)以匹配阵列尺寸。当生产者和消费者的布局偏好不一致时,必须进行布局转换。

本实验旨在回答:**在全网范围内,应该在何处、以何种方式进行布局转换,才能使总的 DRAM 访问代价最小?**

\subsection{代价模型 (Cost Model)}
我们定义总代价为执行代价与转换代价之和:
\begin{equation}
    J = \sum_{v \in V} \text{Cost}_{\text{exec}}(v, L_v) + \sum_{(u,v) \in E} \text{Cost}_{\text{trans}}(L_u, L_v)
\end{equation}

\begin{itemize}
    \item \textbf{执行代价 ($\text{Cost}_{\text{exec}}$)}: 衡量算子在特定布局下访问 DRAM 的效率。模型考虑了 **Burst Efficiency**(带宽利用率)和 **Row Buffer Miss Rate**(行缓存命中率)。非连续访问(Strided Access)会导致高昂的 Row Miss 代价。
    \item \textbf{转换代价 ($\text{Cost}_{\text{trans}}$)}: 衡量在两个不匹配布局之间传输数据的代价。我们支持两种转换模式:
    \begin{itemize}
        \item \textbf{Direct Write}: 生产者按自身顺序写,消费者按自身顺序读(无显式转换,但消费者可能承担高昂的读取代价)。
        \item \textbf{Transform-on-Write}: 生产者利用片上 SRAM 重排数据后写入 DRAM(增加写入代价,但消费者可获得完美的读取顺序)。
    \end{itemize}
\end{itemize}

\subsection{实验流程}
实验代码位于 \texttt{layout\_propagation} 目录下,主要脚本为 \texttt{analyze\_nns\_layout\_v2.py}(优化策略)和 \texttt{analyze\_nns\_layout\_baseline.py}(基准策略)。

\begin{enumerate}
    \item \textbf{网络解析}: 从 \texttt{nn\_dataflow/nns} 读取网络定义。
    \item \textbf{基准测试 (Baseline)}: 强制所有层使用标准的线性布局(NCHW)。这模拟了传统深度学习框架的默认行为。
    \item \textbf{优化测试 (Optimized)}: 运行布局传播算法。
    \begin{itemize}
        \item \textbf{Phase 1 (Propagation)}: 敏感算子(如 Conv)生成硬件友好的分块布局(Blocked Layout),并向邻居传播。
        \item \textbf{Phase 2 (Decision)}: 贪婪策略选择器根据代价模型,为每个节点选择最佳布局,并决定在何处插入转换。
    \end{itemize}
\end{enumerate}

\section{实验结果 (Results)}

我们在四个经典网络上进行了评估。硬件参数设定为:DRAM Row Size = 2KB, Burst Size = 32B, Array Size = 16。

\begin{table}[h]
\centering
\caption{布局优化前后总代价对比}
\label{tab:results}
\begin{tabular}{lcccc}
\toprule
\textbf{网络模型} & \textbf{层数} & \textbf{基准代价 (Baseline)} & \textbf{优化代价 (Optimized)} & \textbf{提升幅度} \\
\midrule
AlexNet   & 19 & 18.63 & 16.83 & 9.66\% \\
VGG Net   & 21 & 31.44 & 30.19 & 3.98\% \\
\textbf{ResNet50}  & 21 & \textbf{45.13} & \textbf{22.39} & \textbf{50.39\%} \\
GoogleNet & 9  & 14.94 & 14.39 & 3.68\% \\
\bottomrule
\end{tabular}
\end{table}

\section{结果分析 (Analysis)}

\subsection{ResNet50 的巨大提升}
ResNet50 取得了超过 50\% 的性能提升,这是本实验最显著的发现。
\begin{itemize}
    \item \textbf{原因}: ResNet 包含大量的残差连接(Residual Connections)。在基准策略(NCHW)下,主分支(Conv)和残差分支(Identity/Conv)的数据往往在汇合点(Add)发生冲突。
    \item \textbf{具体表现}: 在基准测试中,我们观察到如 \texttt{conv2\_br -> conv2\_\{\}\_res} 这样的边产生了高达 \textbf{9.69} 的转换代价。这是因为残差分支通常是 $1 \times 1$ 卷积或直接直连,其在 NCHW 布局下的访问模式与主分支的 Blocked 模式极不匹配,导致了严重的 Row Buffer Thrashing。
    \item \textbf{优化效果}: 布局传播算法成功识别了这一结构,强制残差分支也采用与主分支一致的 Blocked Layout。虽然这可能略微增加了残差分支的写入代价,但彻底消除了汇合点的高昂读取代价。
\end{itemize}

\subsection{线性网络 (VGG/AlexNet) 的平庸表现}
对于 VGG 和 AlexNet,提升幅度较小(<10\%)。
\begin{itemize}
    \item \textbf{原因}: 这些网络是简单的线性链状结构(Conv -> ReLU -> Pool -> Conv)。
    \item \textbf{分析}: 在这种结构中,相邻层通常具有相似的维度特征。虽然 Blocked Layout 在理论上优于 NCHW,但在每一层之间进行 Layout 转换(Transform-on-Write)本身也有开销。对于某些层,直接使用 NCHW(Direct Write)的代价与“转换+优化读取”的代价相差不大。优化器在这种情况下只能通过微调转换位置(例如在 Pooling 层处转换)来获得少量收益。
\end{itemize}

\subsection{GoogleNet 的情况}
GoogleNet 的提升也较小。这主要是因为我们的解析器目前将 Inception 模块简化为了线性序列进行分析,未能完全捕捉其多分支并行的复杂性。如果完整建模 Inception 模块的 4 个分支,预期收益应介于 VGG 和 ResNet 之间。

\section{结论 (Conclusion)}

实验证明,\texttt{layout\_propagation} 提出的自动化布局优化框架是有效的。
\begin{enumerate}
    \item 对于**线性网络**,它能找到不亚于人工设计的布局转换点。
    \item 对于**分支网络**(如 ResNet),它能通过全局协调消除严重的布局冲突,带来显著的性能提升。
\end{enumerate}
这验证了在深度学习编译器中引入“布局传播”机制的必要性,特别是针对具有复杂存储层级和特定访问模式的专用加速器。

\end{document}
