\documentclass[UTF8]{ctexart}
\usepackage{amsmath}
\usepackage{amssymb}
\usepackage{algorithm}
\usepackage{algorithmic}
\usepackage{graphicx}
\usepackage{geometry}
\usepackage{listings}
\usepackage{url}

% Allow line breaks in tt fonts
\newcommand{\code}[1]{\texttt{#1}}

\title{PIM-DLS: 面向存内计算架构的自动化数据流优化框架}
\author{Co-Author}
\date{\today}

\begin{document}

\maketitle

\section{问题形式化}

我们致力于解决在基于 DRAM 的存内计算 (PIM) 架构上优化卷积神经网络 (CNN) 数据流映射的问题。优化目标是通过确定最佳的循环分块 (Loop Tiling)、循环排序 (Loop Ordering) 和数据放置策略,来最小化端到端推理延迟(或能耗)。

\subsection{系统模型}
\textbf{架构模型 ($\mathcal{A}$)}: 硬件被建模为一个具有 $M+1$ 层的分层存储系统,索引从 $m=0$ 到 $M$。
\begin{itemize}
    \item \textbf{计算层级 ($m=0$)}: 代表 PIM 处理单元 (PE) 阵列。我们的模型支持丰富的计算单元类型,包括标量 (Scalar)、SIMD、张量核心 (Tensor Core)、归约树 (Reduction Tree) 以及脉动阵列 (Systolic Array)。每个 PE 的特征由其内部并行度(如 MAC 数量)、归约深度和能耗特性定义。
    \item \textbf{存储层级 ($m > 0$)}: 为了精确研究行缓冲区的复用问题,我们将存储层次细分为:
    \begin{itemize}
        \item \textbf{L1 ($m=1$)}: 全局缓冲区 (Global Buffer),用于片上数据共享。
        \item \textbf{L2 ($m=2$)}: 行缓冲区 (Row Buffer),作为 DRAM 的高速缓存行,是行激活优化的关键层级。
        \item \textbf{L3 ($m=3$)}: DRAM 存储单元 (DRAM Cell),代表片外大容量存储。
    \end{itemize}
    每一层由其容量 $C_m$、读写带宽 $BW_m$ 和旁路 (Bypass) 能力定义。
\end{itemize}

\textbf{工作负载模型 ($\mathcal{W}$)}: 一个 CNN 层由 7 维循环嵌套定义:Batch ($N$)、输出通道 ($K$)、输入通道 ($C$)、输出高度 ($P$)、输出宽度 ($Q$)、卷积核高度 ($R$) 和卷积核宽度 ($S$)。工作负载参数还包括步长 ($str$) 和膨胀系数 ($dil$)。

\subsection{优化问题}
给定 $\mathcal{A}$ 和 $\mathcal{W}$,问题在于找到一个映射配置 $\mathcal{M}$,使得成本函数最小化:
\begin{equation}
    \min_{\mathcal{M}} \text{Cost}(\mathcal{M} | \mathcal{A}, \mathcal{W}) = \max(L_{compute}, L_{memory}, L_{interconnect})
\end{equation}
约束条件包括硬件资源限制(容量、带宽、空间维度)。

\begin{quote}\small
代码凭证: \path{src/pim_optimizer/optimizer.py:PIMOptimizer}; \path{src/pim_optimizer/arch/arch.py:PIMArchitecture}; \path{src/pim_optimizer/workload/workload.py:ConvWorkload}.
\end{quote}

\section{方法概述}

我们提出了一个基于整数线性规划 (ILP) 的自动化框架。核心创新在于其能够对基于 DRAM 的 PIM 的独特性能特征进行建模,特别是“行激活 (Row Activation)”开销。与内存访问成本均匀的传统加速器不同,基于 DRAM 的 PIM 性能对数据布局和访问模式(行命中与未命中)高度敏感。

我们的方法将映射空间分解为三个耦合的子问题:
1.  \textbf{循环分块 (Loop Tiling)}: 确定每个存储层级上每个循环维度的块大小。
2.  \textbf{空间映射 (Spatial Mapping)}: 将特定的循环维度分配给硬件空间资源(PE 行、PE 列、内部并行性)。
3.  \textbf{行激活建模 (Row Activation Modeling)}: 一种基于预计算的方法,用于估算 DRAM 行缓冲区冲突和激活周期。

\begin{quote}\small
代码凭证: \path{src/pim_optimizer/optimizer.py}; \path{src/pim_optimizer/model/row_activation.py}.
\end{quote}

\section{核心方法}

\subsection{决策变量}
我们定义了一组二进制决策变量来表示映射配置。

\subsubsection{分块变量 ($xb$)}
令 $xb_{w, m, s, j, i} \in \{0, 1\}$ 为二进制变量,如果第 $i$ 个候选因子被选中用于存储层级 $m$ 和空间方向 $s$ 上的第 $j$ 个循环维度,则该变量为 1。
\begin{itemize}
    \item $w$: 工作负载索引(用于多层优化)。
    \item $m$: 存储层级索引 ($0 \dots M$)。
    \item $s$: 空间方向索引。
    \begin{itemize}
        \item 对于 PE 层级 ($m=0$): $s \in \{0: H, 1: W, 2: Internal, 3: Temporal\}$。其中 $Internal$ 对应 PE 内部并行性(如 Tensor Core 或 SIMD)。
        \item 对于其他层级 ($m>0$): $s \in \{0: Spatial(dummy), 1: Temporal\}$。
    \end{itemize}
    \item $j$: 循环维度索引 ($0 \dots 6$,对应 $R, S, P, Q, C, K, N$)。
    \item $i$: 除数列表中候选因子的索引。
\end{itemize}

\subsubsection{排列变量 ($xp$)}
令 $xp_{w, m, p, j} \in \{0, 1\}$ 为二进制变量,如果第 $j$ 个循环维度被放置在存储层级 $m$ 的时间循环顺序中的第 $p$ 个优先级,则该变量为 1。其中 $p=0$ 代表最内层循环。

\subsubsection{旁路变量 ($xd$)}
令 $xd_{w, m, t} \in \{0, 1\}$ 为二进制变量,如果数据类型 $t$(0: Input, 1: Weight, 2: Output)存储在存储层级 $m$(即未被旁路),则该变量为 1。

\subsubsection{布局选择变量 ($x_{layout}$)}
令 $x_{layout, w, t, mode} \in \{0, 1\}$ 为二进制变量,用于选择 DRAM 中的数据布局模式。
\begin{itemize}
    \item $mode \in \{sequential, row\_aligned\}$: 分别对应紧凑的顺序布局和对齐到 Row Buffer 的布局。
\end{itemize}

\subsubsection{辅助变量 ($xr, xj$)}
为了处理数据复用和行激活开销,我们引入了辅助变量:
\begin{itemize}
    \item $xr_{w, t, m, m', p}$: 追踪在层级 $m$ 和 $m'$ 之间,优先级 $p$ 处是否存在相关的内层循环。
    \item $xj_{w, t, m, m', j}$: 指示维度 $j$ 在层级 $m$ 和 $m'$ 之间是否包含相关的内层循环(即维度 $j$ 是否导致了数据的重复访问)。
\end{itemize}

\subsubsection{RowBuffer 块选择变量}
针对 Input 张量的特殊优化,我们定义了 $x_{rb\_block, w, i}$ 来选择 Row Buffer 层级的逻辑块大小(Block H/W),以优化滑动窗口的跨行行为。

\begin{quote}\small
代码凭证: \path{src/pim_optimizer/model/variables.py:VariableSet}; \path{src/pim_optimizer/model/variables.py:SpatialDim}.
\end{quote}

\subsection{约束条件}

\subsubsection{维度因式分解}
所有层级和方向上的分块因子的乘积必须等于每个维度 $j$ 的总问题规模 $D_j$。我们使用对数将其线性化:
\begin{equation}
    \sum_{m=0}^{M} \sum_{s} \sum_{i} xb_{w, m, s, j, i} \cdot \log(\text{factor}_{j, i}) = \log(D_j), \quad \forall j
\end{equation}

\subsubsection{唯一性}
对于任何特定的循环位置(由 $m, s, j$ 定义),必须且只能选择一个分块因子:
\begin{equation}
    \sum_{i} xb_{w, m, s, j, i} = 1, \quad \forall m, s, j
\end{equation}

\subsubsection{空间资源约束}
在 PE 层级 ($m=0$),我们强制映射的空间并行性不超过硬件限制。
\begin{itemize}
    \item \textbf{互斥性}: 一个循环维度 $j$ 最多只能映射到一个空间方向(H、W 或 Internal),且因子 $>1$。
    \begin{equation}
        \sum_{s \in \{H, W, Int\}} (1 - xb_{w, 0, s, j, \text{idx}(1)}) \le 1
    \end{equation}
    其中 $\text{idx}(1)$ 是因子 1 的索引。
    \item \textbf{阵列大小}: 映射到方向 $s$ 的因子的乘积必须适应硬件尺寸 $Size_s$:
    \begin{equation}
        \sum_{j} \sum_{i} xb_{w, 0, s, j, i} \cdot \log(\text{factor}_{j, i}) \le \log(Size_s), \quad \forall s \in \{H, W\}
    \end{equation}
\end{itemize}

\subsubsection{缓冲区容量约束}
对于每个存储层级 $m$ 和数据类型 $t$,所需的存储大小不得超过容量 $C_m$。存储大小是相关维度的分块因子的乘积。
\begin{equation}
    \sum_{j \in \text{Relevant}(t)} \sum_{s} \sum_{i} xb_{w, m, s, j, i} \cdot \log(\text{factor}_{j, i}) \le \log(C_m) + M \cdot (1 - xd_{w, m, t})
\end{equation}
如果数据类型被旁路 ($xd=0$),项 $M \cdot (1 - xd_{w, m, t})$ 会放宽约束。

\begin{quote}\small
代码凭证: \path{src/pim_optimizer/model/constraints.py:add_basic_constraints}; \path{src/pim_optimizer/model/expressions.py:build_buffer_utilization}.
\end{quote}

\subsection{行激活建模 (Row Activation Modeling)}
DRAM 行激活 (Row Activation) 是 PIM 性能的关键瓶颈。我们提出了一种混合模型,结合了高精度微观模拟 (Micro-Trace Simulation) 和解析公式 (Analytical Formula)。总开销模型如下:
\begin{equation}
    TotalCost = (1 - z_{aligned}) \cdot Cost_{Seq} + z_{aligned} \cdot Cost_{Aligned} + Cost_{BlockCrossing}
\end{equation}
其中 $z_{aligned} \in \{0, 1\}$ 是选择对齐布局的决策变量。

\subsubsection{顺序布局开销 ($Cost_{Seq}$)}
对于顺序布局 (Sequential Layout),我们根据张量类型采用不同的建模策略:
\begin{itemize}
    \item \textbf{输入张量 (Input)}: 由于滑动窗口导致的复杂访问模式,我们采用查表法。预先计算不同 Tile 尺寸下的平均激活次数 $AvgCost_i$,并将其代入 ILP 公式:
    \begin{equation}
        Cost_{Seq}^{Input} = \sum_{i \in Table} z_{i} \cdot \left[ N_{tiles} + (AvgCost_i \cdot N_{tiles} - N_{tiles}) \cdot ReusePenalty \right]
    \end{equation}
    其中 $z_i$ 是选择第 $i$ 个查表项的二进制变量,$AvgCost_i \cdot N_{tiles}$ 是预计算的总激活次数。公式的物理含义是:基础开销为 $N_{tiles}$(每个 Tile 至少激活一次),额外的跨行开销 $(AvgCost - 1) \cdot N_{tiles}$ 会受到 $ReusePenalty$ 的放大。
    
    \item \textbf{权重/输出张量 (Weight/Output)}: 采用混合成本模型 (Hybrid Cost Model)。以理想的流式访问成本 ($TotalBytes/RowSize$) 为基准,根据 Tile 的对齐性(Alignment)和复用模式(Thrashing)施加惩罚系数,从而区分流式访问和抖动访问。
    \begin{equation}
        Cost_{Seq}^{W/O} = C_{stream} \cdot Reuse \cdot (1 + \mathbb{I}_{penalty})
    \end{equation}
\end{itemize}

\subsubsection{对齐布局开销 ($Cost_{Aligned}$)}
在对齐布局中,每一行数据都被填充至 DRAM Row 边界,消除了行内冲突,但增加了带宽浪费。其开销由解析公式计算:
\begin{equation}
    Cost_{Aligned} = \frac{\text{TotalBytes}_{padded}}{\text{RowSize}} \cdot ReusePenalty
\end{equation}

\subsubsection{输入块跨越开销 ($Cost_{BlockCrossing}$)}
由于卷积的滑动窗口特性,输入 Tile 可能会跨越逻辑块 (Layout Block) 边界。我们使用基于 GCD 的数论方法计算跨越概率 $P_{cross}$:
\begin{equation}
    P_{cross} = \frac{\text{Period} - \lceil (BlockH - TileH + 1) / g \rceil}{\text{Period}}
\end{equation}
其中 $g = \gcd(Step, BlockH)$,$Period = BlockH / g$。最终开销为 $Cost_{BlockCrossing} = N_{tiles} \cdot P_{cross} \cdot C_{penalty}$。

\begin{quote}\small
代码凭证: \path{src/pim_optimizer/model/row_activation.py:precompute_tile_crossing_info}; \path{src/pim_optimizer/model/expressions.py:compute_unique_input_size}.
\end{quote}

\subsection{目标函数}
目标是最小化计算和内存延迟的最大值:
\begin{equation}
    \min \max(L_{compute}, L_{DRAM})
\end{equation}

\textbf{DRAM 延迟 ($L_{DRAM}$)}:
对于每个数据类型 $t$,延迟是数据传输时间和行激活时间的总和:
\begin{equation}
    L_{DRAM, t} = \frac{\text{DataVolume}_t}{BW_{RowBuffer}} + N_{act, t} \times t_{act}
\end{equation}
总 DRAM 延迟由瓶颈数据类型主导:
\begin{equation}
    L_{DRAM} = \max_{t \in \{In, Wgt, Out\}} L_{DRAM, t}
\end{equation}

\begin{quote}\small
代码凭证: \path{src/pim_optimizer/model/objective.py:_build_dram_latency_cycles}.
\end{quote}

\section{实现细节}

该框架使用 Python 实现,并采用 Gurobi 优化器。

\subsection{软件架构}
\begin{itemize}
    \item \textbf{优化器入口 (\path{optimizer.py})}: \path{PIMOptimizer} 类初始化架构和工作负载,然后编排优化过程。
    \item \textbf{模型构建 (\path{model/})}:
        \begin{itemize}
            \item \path{variables.py}: 定义包含所有 Gurobi 变量 ($xb, xp, xd$) 的 \path{VariableSet} 数据类。
            \item \path{constraints.py}: 实现约束生成函数,如 \path{add_basic_constraints} 和 \path{add_buffer_constraints}。
            \item \path{expressions.py}: 包含计算缓冲区利用率和唯一输入大小的逻辑。
        \end{itemize}
    \item \textbf{预计算}: 为了处理行激活的非线性,系统预先计算成本表(通过 \path{row_activation.py}),并将其作为线性系数或查找表注入 ILP 模型。
\end{itemize}

\subsection{执行流程}
1.  \textbf{解析配置}: 加载 \path{arch.yaml} 和 \path{workload.yaml}。
2.  \textbf{生成除数}: 将工作负载维度分解为候选质因数。
3.  \textbf{预计算表}: 运行 \path{precompute_tile_crossing_info} 为所有可能的分块大小生成激活成本。
4.  \textbf{构建 ILP}: 实例化 Gurobi 模型,添加变量和约束。
5.  \textbf{求解}: 运行 \path{model.optimize()},并指定时间限制(默认 300 秒)。
6.  \textbf{提取结果}: 将二进制变量映射回可读的映射参数。

\begin{quote}\small
代码凭证: \path{src/pim_optimizer/cli.py}; \path{src/pim_optimizer/optimizer.py}.
\end{quote}

\section{行激活模块实现}

本节详细介绍 \path{src/pim_optimizer/model/row_activation.py} 中的核心算法实现。该模块负责为 ILP 模型提供精确的 DRAM 行激活开销估算。针对不同类型的张量(Tensor),我们采用了差异化的建模策略。

\subsection{输入张量 (Input Tensor) 建模}
输入张量由于卷积操作的滑动窗口(Sliding Window)特性,其内存访问模式最为复杂。相邻的计算窗口在输入特征图上存在重叠,导致数据复用和行缓冲区冲突(Row Buffer Conflict)难以通过简单的线性公式计算。

\subsubsection{基于 GCD 的跨越分析}
为了解决这一问题,我们实现了 \path{compute_input_block_crossing_count} 函数。该函数利用数论中的最大公约数(GCD)性质,计算滑动窗口跨越逻辑块(Layout Block)边界的精确概率。
\begin{itemize}
    \item \textbf{周期性分析}: 访问模式的周期由 $Period = BlockH / \gcd(Step, BlockH)$ 定义。
    \item \textbf{跨越计数}: 在一个周期内,跨越边界的次数可以通过解析公式直接计算,无需逐个模拟。
    \item \textbf{核分裂 (Kernel Splitting)}: 当卷积核尺寸较大导致需要分多次访问时,函数会分别计算每个子核(Sub-kernel)的跨越情况并累加。
\end{itemize}

\subsubsection{查表法与 AvgCost}
对于输入张量,我们使用 \path{precompute_input_block_crossing_table} 生成查找表。该表存储了不同分块策略下的平均激活开销 ($AvgCost$)。
\begin{itemize}
    \item $AvgCost$ 反映了在特定的 $(P_{tile}, Q_{tile}, C_{tile})$ 组合下,平均每个 Tile 需要多少次额外的行激活。
    \item ILP 模型通过 \path{build_input_block_crossing_expr} 将这些预计算的成本注入到优化目标中。
\end{itemize}

\subsection{权重与输出张量 (Weight/Output) 建模}
与输入张量不同,权重(Weight)和输出(Output)张量的访问模式通常是线性的或分块线性的(Tiled-Linear)。对于这两种张量的顺序布局(Sequential Layout),我们不使用平均值,而是采用基于 GCD 的精确计数方法。

\subsubsection{混合成本模型 (Hybrid Cost Model)}
对于权重和输出张量的顺序布局,代码实现了一种混合成本模型,旨在区分“流式访问 (Streaming)”和“抖动访问 (Thrashing)”两种模式。该模型不直接累加每个 Tile 的跨行开销,而是以理想的流式访问成本为基准,根据 Tile 的对齐情况和复用模式施加惩罚。

\begin{itemize}
    \item \textbf{基准流式成本 ($C_{stream}$)}: 假设数据能够以理想的顺序流过 Row Buffer,其最小激活次数仅取决于数据总量和行大小:
    \begin{equation}
        C_{stream} = \frac{\text{TotalBytes}}{\text{RowSize}}
    \end{equation}
    
    \item \textbf{惩罚系数 ($M_k$)}: 针对每个候选 Tile 尺寸 $k$,根据其是否与 Row Size 对齐以及是否存在复用抖动,确定倍率系数:
    \begin{itemize}
        \item \textbf{流式模式 ($M_k=1$)}: 当 Tile 与 Row 对齐,或者虽然不对齐但没有复用(Reuse=1)时,开销接近基准值。
        \item \textbf{抖动模式 ($M_k=2$)}: 当 Tile 不对齐且存在频繁复用(Reuse>1),或者内部循环导致 Row Buffer 频繁切换(Thrashing)时,开销加倍。
    \end{itemize}

    \item \textbf{复用惩罚 ($ReusePenalty$)}: 该变量由两部分组成,反映了不同层级的无关循环对行激活的影响。
    \begin{itemize}
        \item \textbf{组成部分}:
        \begin{enumerate}
            \item \textbf{L3 内层无关维度}: 位于 L3 (DRAM) 层级之下的无关循环。
            \item \textbf{L2 内层无关维度}: 位于 L2 (Global Buffer) 层级之下的无关循环。
        \end{enumerate}
        \item \textbf{公式}:
        \begin{equation}
            ReusePenalty = \left( \prod_{j \in Irr} \text{Bound}_j^{(1 - x_{j, L3})} \right) \times \left( \prod_{j \in Irr} \text{Bound}_j^{(1 - x_{j, L2})} \right)
        \end{equation}
        其中 $x_{j}=0$ 表示维度 $j$ 为内层(Stationary),$1-x_j=1$ 表示该维度对复用有贡献。
    \end{itemize}

    \item \textbf{最终公式}:
    \begin{equation}
        Cost_{Seq}^{W/O} = \sum_{k} x_{k} \cdot C_{stream} \cdot ReusePenalty \cdot M_k \cdot OuterIrrProduct
    \end{equation}
    其中 $x_k$ 是选择第 $k$ 个 Tile 尺寸的二进制变量。这种建模方式在保证计算效率的同时,能够捕捉到由于不对齐和抖动导致的带宽浪费。
\end{itemize}

\subsection{ILP 集成}
模块通过 \path{_build_layout_conditional_acts} 函数将上述两种成本模型整合。它引入了二进制变量 $z_{aligned}$ 来动态选择“顺序布局”或“对齐布局”,从而允许优化器在“节省空间(紧凑布局)”和“减少延迟(对齐布局)”之间进行权衡。

\section{局限性与范围}
\begin{itemize}
    \item \textbf{工作负载支持}: 当前实现专门针对 \path{ConvWorkload}。支持其他算子(如矩阵乘法、深度卷积)需要扩展工作负载定义和输入大小公式。
    \item \textbf{静态建模}: 延迟模型假设确定性的 DRAM 行为,未考虑动态刷新周期或行缓冲区模型之外的复杂 Bank 冲突。
    \item \textbf{整数因子}: 分块优化仅限于循环维度的整数因子。
\end{itemize}

\begin{quote}\small
代码凭证: \path{src/pim_optimizer/workload/workload.py}; \path{src/pim_optimizer/model/constraints.py}.
\end{quote}

\end{document}