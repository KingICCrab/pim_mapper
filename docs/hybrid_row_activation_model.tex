\documentclass{article}
\usepackage{amsmath}
\usepackage{amssymb}
\usepackage{algorithm}
\usepackage{algpseudocode}
\usepackage{geometry}
\usepackage{graphicx}
\geometry{a4paper, margin=1in}

\title{Hybrid Cost Model for Input Row Activation}
\author{PIM Optimizer Team}
\date{\today}

\begin{document}

\maketitle

\section{Introduction}
Accurately estimating DRAM Row Activations for Input Feature Maps is critical for performance optimization. Simple analytical models (e.g., assuming contiguous access) fail to capture complex interactions between:
\begin{itemize}
    \item \textbf{Tile Alignment}: The relative offset of a tile within a DRAM Row.
    \item \textbf{Loop Order}: The specific traversal order (e.g., Row-Major vs. Block-Major) within a tile.
    \item \textbf{Row Thrashing}: The penalty incurred when a single tile repeatedly crosses DRAM Row boundaries.
\end{itemize}

This document describes a \textbf{Hybrid Cost Model} that combines \textbf{Number Theory (GCD)} for efficient quantity estimation and \textbf{Micro-Trace Simulation} for accurate unit cost calculation.

\section{The Hybrid Approach: Categorization \& Sampling}

The total Input Row Activation cost is derived by categorizing tiles into two states: \textbf{Safe} (aligned within a row) and \textbf{Crossing} (straddling a row boundary).

\begin{equation}
    \text{TotalCost} = N_{safe} \cdot C_{safe} + N_{crossing} \cdot C_{crossing}
\end{equation}

\subsection{1. Cost Determination ($C$)}
We use \textbf{Micro-Trace Simulation} to determine the unit costs.
\begin{itemize}
    \item \textbf{Safe Cost ($C_{safe}$)}: The cost when the tile is perfectly aligned (Offset $= 0$).
    \begin{equation}
        C_{safe} = \text{MicroTrace}(0)
    \end{equation}
    \item \textbf{Crossing Cost ($C_{crossing}$)}: The expected cost when the tile is in the ``Danger Zone''. We sample offsets specifically from the crossing region.
    \begin{equation}
        C_{crossing} = \frac{1}{|\mathcal{O}_{cross}|} \sum_{o \in \mathcal{O}_{cross}} \text{MicroTrace}(o)
    \end{equation}
\end{itemize}

\subsection{2. Quantity Determination ($N$)}
Instead of simulating every tile to check if it crosses, we use \textbf{Number Theory} to calculate the exact counts.

Let $P$ be the Period (Row Buffer Size) and $S$ be the Step (Stride in linear memory).
The ``Danger Zone'' $Z$ is the set of offsets that cause a tile of width $W_{tile}$ to cross a boundary:
\begin{equation}
    Z = [P - W_{tile} + 1, P - 1]
\end{equation}

The sequence of offsets is $o_i = (i \cdot S) \pmod P$. This sequence visits the set of values $\{0, g, 2g, \dots\}$ where $g = \gcd(S, P)$.
The probability of landing in the Danger Zone is:
\begin{equation}
    Prob_{crossing} = \frac{\text{Count of multiples of } g \text{ in } Z}{P/g}
\end{equation}

Thus, the exact quantities are:
\begin{align}
    N_{crossing} &= N_{total} \times Prob_{crossing} \\
    N_{safe} &= N_{total} - N_{crossing}
\end{align}

\section{Micro-Trace Simulation}
To determine the cost of a single tile at a specific offset $o$, we use a lightweight simulation.

\subsection{Strict Ordering (Realistic Hardware)}
We simulate the exact memory access sequence generated by the hardware loop nest (e.g., Row-Major $H, W$).
\begin{algorithm}
\caption{Micro-Trace Simulation (Strict Order)}
\begin{algorithmic}
\State $ActiveRow \gets -1$
\State $Activations \gets 0$
\For{$h \in [0, TileH-1]$}
    \For{$w \in [0, TileW-1]$}
        \State $Addr \gets BaseAddr + o + h \cdot W_{total} + w$
        \State $RowID \gets Addr // RowBufferSize$
        \If{$RowID \neq ActiveRow$}
            \State $Activations \gets Activations + 1$
            \State $ActiveRow \gets RowID$
        \EndIf
    \EndFor
\EndFor
\State \Return $Activations$
\end{algorithmic}
\end{algorithm}

This captures \textbf{Row Thrashing} where a tile straddles a row boundary and the access pattern causes repeated switching (e.g., $A \to B \to A \to B$).

\section{Validation Results}
We compared the Hybrid Model against a full exhaustive simulation of all tiles.

\begin{table}[h]
\centering
\begin{tabular}{|l|l|l|l|l|}
\hline
\textbf{Scenario} & \textbf{Params (H, W, Tile, Stride)} & \textbf{Ground Truth} & \textbf{Hybrid Model} & \textbf{Error} \\ \hline
Aligned & 100, 1024, 3x3, 1 & 3.0000 & 3.0000 & 0.00\% \\ \hline
Misaligned & 224, 224, 3x3, 1 & 1.4324 & 1.4375 & 0.35\% \\ \hline
Large Stride & 224, 224, 3x3, 2 & 1.4324 & 1.4375 & 0.35\% \\ \hline
Large Tile & 224, 224, 7x7, 2 & 2.3119 & 2.3125 & 0.02\% \\ \hline
\end{tabular}
\caption{Validation of Hybrid Cost Model vs. Exhaustive Simulation}
\end{table}

\section{Implementation in ILP}
The ILP optimizer will use this model to precompute a lookup table or cost function:
\begin{enumerate}
    \item For a candidate mapping $(TileH, TileW)$, calculate $E[\text{Cost}]$.
    \item Use $E[\text{Cost}]$ as the coefficient for the Input Row Activation term in the objective function.
    \item This ensures the optimizer penalizes mappings that result in high thrashing probabilities.
\end{enumerate}

\end{document}
