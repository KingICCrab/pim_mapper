\documentclass{article}
\usepackage{amsmath}
\usepackage{amssymb}
\usepackage{tikz}
\usetikzlibrary{shapes.geometric, arrows.meta, positioning, fit, backgrounds}
\usepackage{geometry}
\geometry{a4paper, margin=0.8in}

\title{PIM Optimizer 工作流}
\author{PIM Optimizer}
\date{\today}

\begin{document}

\maketitle

\section{整体工作流程}

\begin{center}
\begin{tikzpicture}[
    node distance=1.5cm,
    box/.style={rectangle, draw, rounded corners, minimum width=3.5cm, minimum height=1cm, align=center, fill=blue!10},
    data/.style={rectangle, draw, minimum width=3cm, minimum height=0.8cm, align=center, fill=green!10},
    file/.style={rectangle, draw, minimum width=2.5cm, minimum height=0.8cm, align=center, fill=yellow!10},
    arrow/.style={-{Stealth[length=3mm]}, thick},
]

% Input files
\node[file] (arch) {arch.yaml\\架构配置};
\node[file, right=1cm of arch] (workload) {workload.yaml\\工作负载};

% Optimizer
\node[box, below=1.2cm of $(arch)!0.5!(workload)$] (optimizer) {PIM Optimizer\\(ILP 求解)};

% Results
\node[data, below=1cm of optimizer] (mapping) {最优 Mapping\\(tile sizes, loop order)};
\node[data, left=0.3cm of mapping] (latency) {预测延迟\\(cycles)};
\node[data, right=0.3cm of mapping] (rowact) {Row Activations\\(预测值)};

% Validation branch
\node[box, below=1.5cm of mapping] (tracegen) {Trace Generator\\生成内存访问 trace};
\node[file, below=1cm of tracegen] (trace) {trace.txt\\LD/ST 序列};
\node[box, below=1cm of trace] (ramulator) {Ramulator2\\DRAM 模拟器};
\node[data, below=1cm of ramulator] (simresult) {模拟结果\\(cycles, row acts)};

% Comparison
\node[box, right=2.5cm of simresult] (compare) {Compare\\ILP vs Simulation};
\node[data, above=1cm of compare] (ilppred) {ILP 预测};

% Arrows
\draw[arrow] (arch) -- (optimizer);
\draw[arrow] (workload) -- (optimizer);
\draw[arrow] (optimizer) -- (mapping);
\draw[arrow] (optimizer) -- (latency);
\draw[arrow] (optimizer) -- (rowact);
\draw[arrow] (mapping) -- (tracegen);
\draw[arrow] (tracegen) -- (trace);
\draw[arrow] (trace) -- (ramulator);
\draw[arrow] (ramulator) -- (simresult);
\draw[arrow] (simresult) -- (compare);
\draw[arrow] (rowact) |- (ilppred);
\draw[arrow] (ilppred) -- (compare);

\end{tikzpicture}
\end{center}

\section{核心模块}

\subsection{1. PIM Optimizer (ILP 求解)}

输入:
\begin{itemize}
    \item \texttt{arch.yaml}: PE array 大小、内存层级、带宽、DRAM 参数
    \item \texttt{workload}: 卷积参数 (N, K, C, P, Q, R, S, stride, dilation)
\end{itemize}

输出:
\begin{itemize}
    \item Tile sizes: 每个维度在每个内存层级的分块大小
    \item Loop permutation: 循环顺序
    \item 预测延迟和 Row Activations
\end{itemize}

\subsection{2. DRAM Latency Model}

每个数据类型的 DRAM 延迟:
\begin{equation}
\text{DRAM\_latency}_t = \frac{\text{mem\_reads}_t}{\text{BW}_{\text{rowbuf}}} + \text{row\_acts}_t \times T_{\text{act}}
\end{equation}

总 DRAM 延迟:
\begin{equation}
\text{DRAM\_latency}_{\text{total}} = \max(\text{DRAM\_latency}_{\text{input}}, \text{DRAM\_latency}_{\text{weight}}, \text{DRAM\_latency}_{\text{output}})
\end{equation}

\subsection{3. 验证流程}

\begin{center}
\begin{tikzpicture}[
    node distance=0.8cm,
    step/.style={rectangle, draw, rounded corners, minimum width=8cm, minimum height=0.8cm, align=left, fill=blue!5},
    arrow/.style={-{Stealth[length=2mm]}, thick},
]

\node[step] (s1) {1. \texttt{print\_mapping.py} - 提取 ILP 结果中的 mapping};
\node[step, below=0.5cm of s1] (s2) {2. \texttt{trace\_generator.py} - 根据 mapping 生成内存访问 trace};
\node[step, below=0.5cm of s2] (s3) {3. \texttt{ramulator\_runner.py} - 运行 Ramulator2 模拟};
\node[step, below=0.5cm of s3] (s4) {4. \texttt{full\_validation.py} - 对比 ILP 预测与模拟结果};

\draw[arrow] (s1) -- (s2);
\draw[arrow] (s2) -- (s3);
\draw[arrow] (s3) -- (s4);

\end{tikzpicture}
\end{center}

\section{文件结构}

\begin{verbatim}
pim_optimizer/
├── src/pim_optimizer/
│   ├── optimizer.py          # 主优化器
│   ├── model/
│   │   ├── variables.py      # ILP 变量定义
│   │   ├── constraints.py    # ILP 约束
│   │   ├── objective.py      # 目标函数 (含 DRAM latency)
│   │   └── row_activation.py # Row Activation 模型
│   ├── arch/                 # 架构定义
│   └── workload/             # 工作负载定义
│
├── validation/dram/
│   ├── print_mapping.py      # 提取并打印 mapping
│   ├── trace_generator.py    # 生成 Ramulator trace
│   ├── ramulator_runner.py   # Ramulator2 接口
│   └── full_validation.py    # 完整验证脚本
│
├── examples/configs/
│   ├── arch.yaml             # 架构配置示例
│   └── conv_workload.yaml    # 工作负载配置示例
│
└── docs/
    └── dram_latency_model.tex  # DRAM 延迟模型文档
\end{verbatim}

\section{使用示例}

\subsection{运行优化}
\begin{verbatim}
# Python API
from pim_optimizer import PIMOptimizer
from pim_optimizer.arch.pim_arch import PIMArchitecture
from pim_optimizer.workload.conv import ConvWorkload

arch = PIMArchitecture.from_yaml('examples/configs/arch.yaml')
workload = ConvWorkload(N=1, K=64, C=64, P=14, Q=14, R=3, S=3)
optimizer = PIMOptimizer(arch)
result = optimizer.optimize([workload])
\end{verbatim}

\subsection{运行验证}
\begin{verbatim}
# 1. 查看 mapping 结果
python validation/dram/print_mapping.py

# 2. 运行完整验证 (需要 Ramulator2)
python validation/dram/full_validation.py
\end{verbatim}

\end{document}
