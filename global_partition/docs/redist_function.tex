%==============================================================================
% 重分布代价函数 Redist 详细说明
% 作为 global_partition_paper.tex 的补充章节
%==============================================================================

\subsection{重分布代价函数 $\text{Redist}(l, c_i, c_j)$ 详细定义}
\label{subsec:redist-function}

重分布代价函数 $\text{Redist}(l, c_i, c_j)$ 描述了当第 $l$ 层选择分区方案 $c_i$、第 $l+1$ 层选择分区方案 $c_j$ 时,层间数据重分布所需的通信代价。该函数是全局分区优化目标函数的关键组成部分。

\subsubsection{重分布类型判定}

首先定义判定函数 $\text{Type}(c_i, c_j)$,根据两个相邻分区方案确定重分布类型:

\begin{equation}
\text{Type}(c_i, c_j) = 
\begin{cases}
\text{NONE} & \text{if } c_i = c_j \text{ 且无维度变化} \\[2pt]
\text{LOCAL} & \text{if } K_i = K_j \text{ 且 } H_i = H_j \text{ 且 } W_i = W_j \\[2pt]
\text{ALL\_REDUCE} & \text{if } C_i > 1 \text{ (INPP分区需归约)} \\[2pt]
\text{ALL\_GATHER} & \text{if } K_i > 1 \text{ 且 } K_j = 1 \\[2pt]
\text{SCATTER} & \text{if } K_i = 1 \text{ 且 } K_j > 1 \\[2pt]
\text{ALL\_TO\_ALL} & \text{otherwise (一般情况)}
\end{cases}
\label{eq:redist-type}
\end{equation}

其中 $K_i, H_i, W_i, C_i$ 分别表示方案 $c_i$ 在输出通道、高度、宽度、输入通道维度上的分区因子。

\subsubsection{通信量计算}

设第 $l$ 层的输出数据总量为 $D_l = N \times K_l \times H_l \times W_l \times w$,其中 $w$ 为数据位宽(字节)。各重分布类型的通信量如下:

\paragraph{NONE (无需重分布)}
当上下游分区完全兼容时,数据天然分布在正确位置,无需通信:
\begin{equation}
V_{\text{NONE}} = 0
\end{equation}

\paragraph{LOCAL (本地重排)}
分区因子相同但数据布局需要本地调整:
\begin{equation}
V_{\text{LOCAL}} = \alpha_{\text{local}} \cdot D_l, \quad \alpha_{\text{local}} \approx 0.01
\end{equation}
这对应 SRAM 内的数据搬移,代价远小于片间通信。

\paragraph{ALL\_GATHER (收集)}
当上游按 $K$ 分区、下游不分区或使用不同维度分区时,需要收集分布式数据:
\begin{equation}
V_{\text{ALL\_GATHER}} = D_l \cdot \frac{n_i - 1}{n_i}
\end{equation}
其中 $n_i = \prod_d f_{i,d}$ 是方案 $c_i$ 使用的总节点数。

\paragraph{SCATTER (分发)}
当上游不分区而下游需要分区时,需要将完整数据分发到各节点:
\begin{equation}
V_{\text{SCATTER}} = D_l \cdot \frac{n_j - 1}{n_j}
\end{equation}

\paragraph{ALL\_TO\_ALL (全交换)}
当两侧分区维度或因子不同时,需要全交换通信:
\begin{equation}
V_{\text{ALL\_TO\_ALL}} = D_l \cdot \left(1 - \frac{1}{\max(n_i, n_j)}\right)
\end{equation}

\paragraph{ALL\_REDUCE (归约)}
当上游使用 INPP (输入通道) 分区时,每个节点计算的是部分和,需要跨节点归约:
\begin{equation}
V_{\text{ALL\_REDUCE}} = 2 \cdot D_l \cdot \frac{n_C - 1}{n_C}
\label{eq:allreduce}
\end{equation}
其中 $n_C$ 是 INPP 分区因子,系数 2 对应 Ring AllReduce 的 reduce-scatter 和 all-gather 两个阶段。

\subsubsection{时间代价模型}

考虑片上网络 (NoC) 拓扑,重分布时间代价为:
\begin{equation}
T_{\text{redist}} = \frac{V}{B_{\text{NoC}}} \cdot h_{\text{avg}}
\end{equation}
其中 $B_{\text{NoC}}$ 是 NoC 带宽,$h_{\text{avg}}$ 是平均跳数。

对于 Mesh 拓扑 ($\sqrt{n} \times \sqrt{n}$):
\begin{equation}
h_{\text{avg}}^{\text{mesh}} = \frac{2\sqrt{n}}{3}
\end{equation}

对于 Crossbar 拓扑:
\begin{equation}
h_{\text{avg}}^{\text{crossbar}} = 1
\end{equation}

\subsubsection{完整的 Redist 函数定义}

综合以上分析,$\text{Redist}(l, c_i, c_j)$ 的完整定义为:

\begin{equation}
\boxed{
\text{Redist}(l, c_i, c_j) = 
\begin{cases}
0 & \text{Type} = \text{NONE} \\[4pt]
\displaystyle \frac{\alpha_{\text{local}} \cdot D_l \cdot h_{\text{avg}}}{B_{\text{NoC}}} & \text{Type} = \text{LOCAL} \\[8pt]
\displaystyle \frac{D_l \cdot (n_i - 1) \cdot h_{\text{avg}}}{n_i \cdot B_{\text{NoC}}} & \text{Type} = \text{ALL\_GATHER} \\[8pt]
\displaystyle \frac{D_l \cdot (n_j - 1) \cdot h_{\text{avg}}}{n_j \cdot B_{\text{NoC}}} & \text{Type} = \text{SCATTER} \\[8pt]
\displaystyle \frac{D_l \cdot (\max(n_i,n_j) - 1) \cdot h_{\text{avg}}}{\max(n_i,n_j) \cdot B_{\text{NoC}}} & \text{Type} = \text{ALL\_TO\_ALL} \\[8pt]
\displaystyle \frac{2 \cdot D_l \cdot (n_C - 1) \cdot h_{\text{avg}}}{n_C \cdot B_{\text{NoC}}} & \text{Type} = \text{ALL\_REDUCE}
\end{cases}
}
\label{eq:redist-full}
\end{equation}

\subsubsection{能耗代价模型}

对于能耗优化场景,重分布能耗为:
\begin{equation}
E_{\text{redist}} = V \cdot h_{\text{avg}} \cdot e_{\text{hop}}
\end{equation}
其中 $e_{\text{hop}}$ 是每字节每跳的能耗 (pJ/byte/hop)。典型值约 1-5 pJ/byte/hop。

\subsubsection{分区传播约束的影响}

\textbf{关键观察}:由于分区传播约束 $K_l = C_{l+1}$,当第 $l$ 层选择 K-分区 (OUTP) 时,其输出数据天然按 C 维度分布在第 $l+1$ 层的输入侧。

\begin{proposition}[分区传播降低重分布代价]
设连续 $k$ 层都选择相同的 OUTP 分区因子 $f$,则这 $k$ 层之间的总重分布代价为:
\begin{equation}
\sum_{i=0}^{k-2} \text{Redist}(l+i, c, c) = 0
\end{equation}
即相邻层选择兼容分区可完全消除层间重分布代价。
\end{proposition}

这一性质是全局优化相比贪心方法的核心优势:通过考虑层间依赖,优化器可以找到使总重分布代价最小化的分区方案序列。

\subsubsection{实现注意事项}

在实际实现中,Redist 函数的计算需要:

\begin{enumerate}
    \item \textbf{层输出尺寸}:$D_l = N \times K_l \times H_l \times W_l \times w$
    \item \textbf{分区因子提取}:从方案 $c_i, c_j$ 中提取各维度分区因子
    \item \textbf{NoC 参数}:拓扑类型、带宽、能耗参数
    \item \textbf{类型判定}:按式 \eqref{eq:redist-type} 判定重分布类型
    \item \textbf{代价计算}:按式 \eqref{eq:redist-full} 计算代价
\end{enumerate}

\begin{lstlisting}[language=Python, caption=Redist 函数的 Python 实现框架]
def redist(layer_idx, choice_i, choice_j, layers, noc_config):
    """
    计算层间重分布代价
    
    Args:
        layer_idx: 当前层索引
        choice_i: 当前层分区方案
        choice_j: 下一层分区方案
        layers: 网络层列表
        noc_config: NoC 配置 (bandwidth, topology, energy_per_hop)
    
    Returns:
        重分布代价 (时间或能耗)
    """
    # 1. 获取输出数据量
    layer = layers[layer_idx]
    D = layer.nofm * layer.hofm * layer.wofm * word_size
    
    # 2. 提取分区因子
    K_i = choice_i.get_factor(PartDim.OUTP)
    C_i = choice_i.get_factor(PartDim.INPP)
    H_i = choice_i.get_factor(PartDim.OFMP_H)
    W_i = choice_i.get_factor(PartDim.OFMP_W)
    n_i = choice_i.total_nodes
    
    K_j = choice_j.get_factor(PartDim.OUTP)
    n_j = choice_j.total_nodes
    
    # 3. 判定重分布类型
    if C_i > 1:
        redist_type = 'ALL_REDUCE'
    elif K_i > 1 and K_j == K_i and H_i == H_j and W_i == W_j:
        redist_type = 'NONE'  # 或 'LOCAL'
    elif K_i > 1 and K_j == 1:
        redist_type = 'ALL_GATHER'
    elif K_i == 1 and K_j > 1:
        redist_type = 'SCATTER'
    else:
        redist_type = 'ALL_TO_ALL'
    
    # 4. 计算代价
    h_avg = compute_avg_hops(noc_config.topology, max(n_i, n_j))
    B = noc_config.bandwidth
    
    if redist_type == 'NONE':
        return 0
    elif redist_type == 'LOCAL':
        return 0.01 * D * h_avg / B
    elif redist_type == 'ALL_GATHER':
        return D * (n_i - 1) / n_i * h_avg / B
    elif redist_type == 'SCATTER':
        return D * (n_j - 1) / n_j * h_avg / B
    elif redist_type == 'ALL_TO_ALL':
        n = max(n_i, n_j)
        return D * (n - 1) / n * h_avg / B
    elif redist_type == 'ALL_REDUCE':
        return 2 * D * (C_i - 1) / C_i * h_avg / B
\end{lstlisting}

\subsubsection{与 ILP 目标函数的结合}

在 ILP 公式 \eqref{eq:ilp} 中,Redist 函数作为系数预计算并存储:

\begin{equation}
\min \ \sum_{l=0}^{L-1}\sum_{c\in\mathcal{C}_l} x_{l,c}\cdot \text{Comp}(l,c)
+\sum_{l=0}^{L-2}\sum_{c_i\in\mathcal{C}_l}\sum_{c_j\in\mathcal{C}_{l+1}} y_{l,c_i,c_j}\cdot \underbrace{\text{Redist}(l,c_i,c_j)}_{\text{预计算常数}}
\end{equation}

由于 $\text{Redist}(l,c_i,c_j)$ 仅依赖于网络结构和硬件参数,可以在构建 ILP 模型前完成所有 $O(L \cdot m^2)$ 个代价值的计算,其中 $m$ 是每层的平均候选方案数。

%==============================================================================
\subsection{Layout Transform 融合优化}
\label{subsec:layout-fusion}

除了分区重分布,层间可能还存在 \textbf{Layout 变换} (如 NCHW $\leftrightarrow$ NHWC)。
如第 \ref{sec:ilp-optimization} 节所述,单独的 Layout Transform 算子会导致额外的 DRAM $\rightarrow$ DRAM 数据移动。

优化策略是将 Layout 变换融合到 Memory Hierarchy 的数据搬移过程中:

\begin{equation}
\text{Redist}_{\text{fused}}(l, c_i, c_j, \text{layout}_i, \text{layout}_j) = 
\text{Redist}(l, c_i, c_j) + \text{LayoutCost}(\text{layout}_i, \text{layout}_j)
\end{equation}

其中当变换融合到 DRAM $\rightarrow$ SRAM 搬移时:
\begin{equation}
\text{LayoutCost}_{\text{fused}} = 0 \quad \text{(无额外代价)}
\end{equation}

而单独执行时:
\begin{equation}
\text{LayoutCost}_{\text{separate}} = \frac{2 \cdot D_l}{B_{\text{DRAM}}} \quad \text{(读+写)}
\end{equation}

典型节省约 47.5\% 能耗(详见 Layout Propagation 模块分析)。
