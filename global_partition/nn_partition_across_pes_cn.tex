\subsection{跨处理单元的神经网络分区}
\label{sec:nn-partition}

处理单元(PE)阵列包含多个PE,每个PE拥有独立的本地存储。除了在不同PE上处理不同的神经网络或层之外,我们还可以将大型神经网络层划分到多个PE上并行处理。本节首先给出不同分区方案的分类(如图~\ref{fig:partition-taxonomy}所示),然后系统地探索并找到最优方案。

\subsubsection{分区方案分类}

\begin{figure}[htbp]
\centering
\includegraphics[width=0.9\linewidth]{figures/partition_taxonomy.pdf}
\caption{跨PE的层分区方案分类。(a) 批量分区将不同图像分配到各PE。(b) 特征图分区划分空间区域。(c) 输出分区划分输出通道。(d) 输入分区划分输入通道并需要规约操作。}
\label{fig:partition-taxonomy}
\end{figure}

\textbf{批量分区(数据并行):}最简单的方案是使用多个PE并行处理多个输入图像,有效地将一个batch分配到各PE~\cite{data-parallel}。虽然这对吞吐量有利,但要求神经网络模型在每个PE中复制,这对大型网络是显著的容量挑战。此外,并行度受batch大小限制。对于延迟敏感的实时应用,该方案吸引力较低,因为它不能改善单张图像的推理延迟。

\textbf{特征图(空间)分区:}如果特征图较大(如$112 \times 112$),可以将其分割成更小的块,如图~\ref{fig:partition-taxonomy}(b)所示。更小的特征图块能更好地适配PE阵列,减少行固定数据流中的折叠需求~\cite{eyeriss}。此外,如果卷积层的输入和输出特征图使用相同的空间分区,由于2D卷积的局部性,大多数数据访问将是PE本地的。但是,卷积核权重需要在所有PE间复制。

\textbf{输出分区:}由于每层通常有多个输出特征图,可以将输出特征图分配到各PE。例如,将$K$个输出特征图分成$p$组,每个PE处理$K/p$个。由于每个输出特征图使用不同的卷积核,权重可以完全分区。但由于所有输入特征图都贡献到所有输出特征图,所有输入特征图必须发送到所有PE,需要远程PE访问。

\textbf{输入分区:}类似输出分区,也可以将$C$个输入特征图分配到各PE。区别在于计算发生的位置。然而,输出特征图的访问同时产生读写流量,因此比输入特征图访问更关键。因此,使用避免输出特征图远程访问的输出分区,优于避免输入特征图远程访问的输入分区。\textbf{重要的是,输入分区需要All-Reduce操作来聚合各PE的部分和},这会引入显著的通信开销。

表~\ref{tab:partition-comparison}总结了各分区方案的特点。

\begin{table}[htbp]
\centering
\caption{分区方案比较}
\label{tab:partition-comparison}
\begin{tabular}{lcccc}
\toprule
\textbf{方案} & \textbf{权重} & \textbf{输入特征图} & \textbf{输出特征图} & \textbf{需要规约} \\
\midrule
批量 (BATP) & 复制 & 分区 & 分区 & 否 \\
空间 (OFMP) & 复制 & 分区 & 分区 & 否 \\
输出 (OUTP) & 分区 & 广播 & 分区 & 否 \\
输入 (INPP) & 分区 & 分区 & \textbf{All-Reduce} & \textbf{是} \\
\bottomrule
\end{tabular}
\end{table}

\subsubsection{混合分区方案}

先前工作~\cite{neurocube}使用简单启发式来最小化远程PE访问:卷积层使用特征图分区,全连接层使用输出分区。这个启发式不一定能带来最佳的性能和能效。除了远程访问数量,还应考虑分区方案对总内存访问数和PE寄存器文件中数据重用的影响。

因为特征图分区会切分特征图,每个PE需要加载相同的卷积核到寄存器,而只在更小的特征图块上重用。相反,输出分区将整个特征图保持在一个PE中,一次加载卷积核并在整个特征图上使用,从而获得更高的权重重用率。

因此,我们考虑一个\textbf{混合分区方案},在特征图分区的优势(最小化远程访问)和输出分区的优势(更好的片上数据重用以最小化总DRAM访问)之间取得平衡。这种混合方案也符合Neurocube启发式的意图:
\begin{itemize}
    \item 对于常见CNN的前几层卷积,大特征图的访问占主导,权重重用不显著,应主要使用\textbf{特征图分区}。
    \item 对于全连接层,卷积核权重远大于特征图,\textbf{输出分区}更优以最大化权重重用。
\end{itemize}

\subsubsection{成本模型}

为找到最优分区方案,我们提出一个基于总内存访问能耗的成本模型:
\begin{equation}
E_{\text{access}} = A_{\text{DRAM}} \times e \times (1 + \beta \cdot r)
\label{eq:access-cost}
\end{equation}
其中$e$是一次本地DRAM访问的能耗,$\beta$是远程PE访问的能耗惩罚因子,$r$是远程访问的比例。$A_{\text{DRAM}}$是总主存访问次数。特征图分区最小化$r$但导致更大的$A_{\text{DRAM}}$,而输出分区通过牺牲$r$来获得更小的$A_{\text{DRAM}}$。

\textbf{INPP All-Reduce成本:}当使用输入通道分区(INPP)时,每个PE计算部分和,必须通过All-Reduce操作聚合:
\begin{equation}
E_{\text{INPP}} = D_{\text{output}} \times \frac{2(p-1)}{p} \times e_{\text{comm}}
\label{eq:inpp-cost}
\end{equation}
其中$D_{\text{output}}$是输出数据量,$p$是INPP分区因子,$e_{\text{comm}}$是单位字节通信能耗。因子$\frac{2(p-1)}{p}$来自Ring All-Reduce算法(Reduce-Scatter后接All-Gather)。

\subsubsection{层类型特定约束}

不同层类型有不同的有效分区维度,如表~\ref{tab:layer-constraints}所示。

\begin{table}[htbp]
\centering
\caption{不同层类型的有效分区维度}
\label{tab:layer-constraints}
\begin{tabular}{lccccl}
\toprule
\textbf{层类型} & \textbf{OUTP} & \textbf{OFMP} & \textbf{BATP} & \textbf{INPP} & \textbf{说明} \\
\midrule
卷积层 & \checkmark & \checkmark & \checkmark & \checkmark & 所有维度有效 \\
全连接层 & \checkmark & $\times$ & \checkmark & \checkmark & OFMP无效($H{=}W{=}1$) \\
池化层 & \checkmark & \checkmark & \checkmark & $\times$ & INPP无效(无跨通道计算) \\
逐元素层 & \checkmark & \checkmark & \checkmark & $\times$ & 无规约,透传分区 \\
\bottomrule
\end{tabular}
\end{table}

\subsubsection{跨层传播}

每个神经网络层都可以使用特征图分区和输出分区的组合来并行化。\textbf{相邻层是耦合的,因为前一层的分区方案决定了下一层输入特征图的布局}(见图~\ref{fig:partition-taxonomy})。

分区方案的跨层传播包括:
\begin{enumerate}
    \item \textbf{OUTP (K) $\rightarrow$ C:}第$l$层的输出通道分区成为第$l{+}1$层的输入通道分布。如果第$l$层使用OUTP=$p$且第$l{+}1$层使用INPP=$p$,分布匹配,无需重分布。
    \item \textbf{OFMP (H,W) $\rightarrow$ 空间:}空间分区传播到下一层的输入空间分布,需要考虑卷积边界的halo区域。
    \item \textbf{BATP (N) $\rightarrow$ 批量:}批量分区通常跨层匹配,无需重分布。
    \item \textbf{INPP (C) $\rightarrow$ 规约:}输入分区需要层内All-Reduce,不影响下一层的布局。
\end{enumerate}

\subsubsection{分区探索}

假设有$L$层和每层$C$种分区组合,需要考虑$C^L$种场景的总内存访问能耗成本。由于$L$可能是20到100,$C$可能是4到8,该成本是禁止性的。

我们利用三种策略来降低分区探索的难度:

\textbf{策略1:贪心算法。}使用贪心算法探索第$i$层的分区选项时不回溯,假设最优方案中前$i-1$层的分区独立于后续层。第一个卷积层假设只使用特征图分区。这将选择数减少到$C \times L$,在常见情况下大约几千种。

\textbf{策略2:布局传播。}基于规约分析,可以减少分区决策变量:
\begin{itemize}
    \item \textbf{有规约的算子}(卷积、全连接、池化)$\rightarrow$ 布局敏感 $\rightarrow$ 需要独立分区决策
    \item \textbf{无规约的算子}(逐元素操作、ReLU、Add)$\rightarrow$ 布局不敏感 $\rightarrow$ 可透传上游分区
\end{itemize}

这允许将决策变量从$L$层减少到$G$组($G \leq L$),其中同组的层共享相同分区:
\begin{equation}
\text{原始变量:} L \times C \quad \Rightarrow \quad \text{优化后:} G \times C
\end{equation}

例如,在ResNet-50中,$G/L \approx 78\%$,减少22\%的决策变量。在ResNet-152中,减少可达24\%。

\textbf{策略3:动态规划/整数线性规划。}对于中等规模问题,使用动态规划或整数线性规划(ILP)求解全局最优:
\begin{equation}
\min \sum_{l} E_{\text{compute}}(l) + \sum_{l} E_{\text{redistribute}}(l, l{+}1)
\end{equation}
需满足分区因子约束、总节点数约束和层类型约束。

动态规划方法的复杂度为$O(L \times C^2)$,对于典型网络是可处理的。
